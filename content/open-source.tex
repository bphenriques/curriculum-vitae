\cvsection{Open Source}

\begin{cventries}
  \openSourceEntry
    {\href{https://github.com/bphenriques/dotfiles}{bphenriques/dotfiles}}
    {April 2020 - Present}
    {
    \begin{openSourceDescription}
    Declarative definition of NixOS and MacOS systems using \href{https://nixos.org/}{Nix} and \href{https://nixos.wiki/wiki/Flakes}{Nix Flakes} for reproducibility.
    \end{openSourceDescription}
    \vspace{4mm}
    \skilltaglist{\textbf{Nix},Nix Flakes,NixOS,Nix-Darwin,Home-Manager,Shell Scripting}
    }

  \vspace{2mm}

  \openSourceEntry
    {\href{https://github.com/bphenriques/self-hosted}{bphenriques/self-hosted}}
    {December 2023 - Present}
    {
    \begin{openSourceDescription}
    Specification and documentation of the services I am self-hosting behind my personal domain name purchased in Cloudflare.
    \end{openSourceDescription}
    \vspace{4mm}
    \skilltaglist{Docker,DNS,Cloudflare,Traefik,Tailscale,Shell Scripting}
    }

  \vspace{2mm}

\openSourceEntry
    {\href{https://github.com/bphenriques/scala-example-slack-bot}{bphenriques/scala-example-slack-bot}}
    {October 2023 - November 2023}
    {   
    \begin{openSourceDescription}
    Proof-of-concept on how to bridge \href{https://github.com/http4s/http4s}{http4s} with \href{https://github.com/slackapi/java-slack-sdk}{Slack's Java SDK} to build interactive Slack applications:
    \end{openSourceDescription}
    \vspace{8mm}
    \begin{openSourceItems}
        \item Http4s middleware that uses Slack's SDK to authenticate requests and bridge requests/responses.
        \item Functional API to retrieve and parse the state of interactions.
    \end{openSourceItems}
    \vspace{4mm}
    \skilltaglist{Slack API,\textbf{Cats},\textbf{Cats Effect},Http4s}
    }

  \vspace{2mm}  

  \openSourceEntry
    {(archived) \href{https://github.com/bphenriques/knowledge-base}{bphenriques/knowledge-base} and \href{https://github.com/bphenriques/explorer-hugo-theme}{bphenriques/explorer-hugo-theme}}
    {May 2020 - July 2023}
    {    
    \begin{openSourceItems}
        \item Public Zettelkasten-based notes using \href{https://ox-hugo.scripter.co/}{ox-hugo}.
        \item Tag-based navigation.
        \item Backlinks and navigation graph using information extracted from \href{https://www.orgroam.com/}{org-roam}.
    \end{openSourceItems}
    \vspace{4mm}
    \skilltaglist{Knowledge Base,Digital Garden,Emacs,\href{https://www.orgroam.com/}{org-roam},\href{https://orgmode.org/}{org-mode},\href{https://gohugo.io/}{Hugo},\href{https://ox-hugo.scripter.co/}{ox-hugo},Javascript,HTML,SCSS,Github Actions, Nix}
    }

  \vspace{2mm}

%\openSourceEntry
%    {\href{https://github.com/bphenriques/curriculum-vitae}{bphenriques/curriculum-vitae}}
%    {Oct. 2022 - Present}
%    {   
%    \begin{openSourceDescription}
%    The source-code of this curriculum vitae.
%    \end{openSourceDescription}
%    \vspace{4mm}
%    \skilltaglist{LaTeX, Github Actions}
%    }
% \vspace{2mm}
    
\openSourceEntry
    {(archived) \href{https://github.com/bphenriques/knowledge-base}{bphenriques/RetroSync}}
    {February 2023}
    {   
    \begin{openSourceDescription}
    Experimental shell-based GUI application to sync save games between handhelds.
    \end{openSourceDescription}
    \vspace{4mm}
    \skilltaglist{Bash,rclone,Shell Scripting}
    }

  \vspace{2mm}

\openSourceEntry
    {Code Examples}
    {Present}
    {   
    \begin{openSourceDescription}
    My Github profile contains code samples across different repositories. The tech stack and quality will vary depending on what I was working at the time and the experience I had at the time:
    \end{openSourceDescription}
    \vspace{8mm}
    \begin{openSourceItems}
        \item \href{https://github.com/bphenriques?tab=repositories&q=interview-challenge&type=&language=&sort=}{Repositories starting with "interview-challenge"} are resolutions to old interview challenges.
        \item \href{https://github.com/bphenriques/scala-exercises}{bphenriques/scala-exercises} contains the resolutions to Scala exercises (some use \href{https://scala-cli.virtuslab.org/}{scala-cli}).
    \end{openSourceItems}
    \vspace{4mm}
    \skilltaglist{\textbf{Scala},\textbf{Kotlin},\textbf{Docker},\textbf{Cats Effect},\textbf{FS2},PostgreSQL,Domain Driven Design,Hexagonal Architecture}
    }
    
\end{cventries}
